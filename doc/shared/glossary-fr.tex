
%%%%%%%%%%%%%%%%%%%%%%%%%%%%%%%%%%%%%%%%%%%%%%%%%%%%%%%%%%%%%%%%%%%%%%%%%%%%%%%
%%                           Glossary                                        %%
%%%%%%%%%%%%%%%%%%%%%%%%%%%%%%%%%%%%%%%%%%%%%%%%%%%%%%%%%%%%%%%%%%%%%%%%%%%%%%%

\newglossaryentry{halo}
		{name={halo}, 
		description={C'est l'ensemble des icônes qui entourent un objet quand on fait un \rc dessus}}
\newglossaryentry{script}
		{name={script}, 
		description={Les scripts sont les programmes associés à chaque objet. On peut afficher la liste des scripts d'un objets dans son \keyword{viewer}, dans la catégorie \important{Scripts}}
		plural={scripts}}
\newglossaryentry{viewer}
		{name={visualisateur}, 
		description={Le visualisateur est la zone, spécifique à chaque objet, qui contient toutes les briques permettant la programmation. On l'affiche grâce à \icon[ \oe il ]{eye} du halo de l'objet. Le visualisateur est divisé en \emph{Catégories} qui servent à organiser les briques}}
\newglossaryentry{heading}
		{name={cap}, 
		description={Le cap d'un objet est son orientation (en degrés) par rapport à sa \emph{Direction avant} : un cap de 0 équivaut à une orientation vers \og l'avant \fg de l'objet. Quand on fait tourner un peu l'objet avec \icon[tourner]{rotate}, on change le cap. La flêche verte qui apparait au centre de l'objet indique la direction avant. On peut la changer en cliquant dessus \important{en maintenant la touche \textit{Majuscule} appuyée}}}
\newglossaryentry{parameter}
		{name={paramètre}, 
		description={Un paramètre d'une fonction est une option que l'on passe à la fonction et que l'on peut modifier}}
\newglossaryentry{world}
		{name={Monde}, 
		description={Le  monde  est l'ensemble du bureau de \appName. C'est un objet similaire aux autres~: on peut afficher son \keyword{halo}, son \keyword{viewer}, lui associer des \keywordpl{script}...}}
\newglossaryentry{photoresistor}
		{name={photo-résistance}, 
		description={Une photo-résistance est un petit capteur de lumière dont la résistance varie avec l'intensité lumineuse qu'il reçoit}}
\newglossaryentry{passive sensor}
		{name={capteur passif},
		description={Un \emph{capteur passif} est un capteur qui n'a pas besoin d'une source externe d'énergie pour fonctionner. Par exemple, la \keyword{photoresistor} ou la thermistance}}
\newglossaryentry{variable}
		{name={variable}, 
		plural={variables}, 
		description={Une variable est une étiquette que l'on peut créer dans un objet, et qui va faire \textit{référence} à un autre objet, à un texte, à une valeur numérique... ça permet essentiellement de manipuler ou d'utiliser cet autre objet ou valeur dans un script}}
%%%%%%%%%%%%%%%%%%%%%%%%%%%%%%%%%%%%%%%%%%%%%%%%%%%%%%%%%%%%%%%%%%%%%%%%%%%%%%%
